\documentclass[addpoints]{exam}
\usepackage[usenames,dvipsnames]{xcolor}
\usepackage{titling}
\usepackage{listings}
\usepackage{multicol}
\usepackage[letterpaper]{geometry}
\geometry{verbose,tmargin=0.75in,bmargin=0.75in,lmargin=0.75in,rmargin=0.75in}

\newcommand{\subtitle}[1]{%
  \posttitle{%
    \par\end{center}
    \begin{center}\large#1\end{center}
    \vskip0.5em}%
}

\lstset{%set Code listings styles
	language=Java, % program language for keywords and comments styles
	basicstyle=\small, %font size and style
	identifierstyle=\color{black}, %variable name style
	stringstyle=\ttfamily, %string style
	keywordstyle=\color{blue}\bfseries, %language keyword style
	commentstyle=\color{blue!75}\itshape, %commentstyle
	breaklines=true,  % sets automatic line breaking
	breakatwhitespace=false,   %break line not just at whitespaces
	showspaces=false,
	showstringspaces=false,
}

\pagestyle{headandfoot}
\runningheadrule
\firstpageheadrule
\firstpageheader{CSCI 305}{May 2, 2018 -- Final Exam}{Name:\enspace\hspace{1.5in}}
\runningheader{CSCI 305}{May 2, 2018 -- Final Exam}{Name:\enspace\hspace{1.5in}}
\firstpagefooter{}{Page \thepage\ of \numpages}{}
\runningfooter{}{Page \thepage\ of \numpages}{}
% \printanswers

\begin{document}
\begin{coverpages}
\begin{center}
  {\makebox[3.5in]{\hrulefill}\\
  \vspace{1.5em}
  {\Large MONTANA STATE UNIVERSITY}\\
  \vspace{.5em}
  PROGRAMMING LANGUAGES\\
  \vspace{.5em}
  Final\\
  \vspace{.5em}
  SPRING, 2018
  \\
  \vspace{1em}
  \makebox[3.5in]{\hrulefill}\\
  \par}
  \vspace{0.5in}
  \makebox[3.5in]{Name:\enspace\hrulefill}

  \vspace{0.2in}

  \makebox[3.5in]{Student Id:\enspace\hrulefill}
  \vspace{0.2in}  

  \setlength{\doublerulesep}{0.1in}
  \gradetable[v][questions]
\end{center}

\end{coverpages}

\begin{questions}
\question[30]
 Mark each question as either True (T) or False (F).\\
 \begin{parts}
  \part
  \begin{oneparcheckboxes}
    \choice T
    \choice F
  \end{oneparcheckboxes}
  A higher order function is one that accepts functions as input or returns a function as output.\\
  \part
  \begin{oneparcheckboxes}
    \choice T
    \choice F
  \end{oneparcheckboxes}
  Scheme's syntax avoids the dangling-else problem by having every if-then-else enclosed in its own set of parentheses.\\
  \part
  \begin{oneparcheckboxes}
    \choice T
    \choice F
  \end{oneparcheckboxes}
  In Scheme, any value that is not considered as False is interpreted as True.\\
  \part
  \begin{oneparcheckboxes}
    \choice T
    \choice F
  \end{oneparcheckboxes}
  Eager evaluation is a technique that can make it easy to avoid unnecessary computation.\\
  \part
  \begin{oneparcheckboxes}
    \choice T
    \choice F
  \end{oneparcheckboxes}
  Scheme uses backtracking in order to search for a solution to a function call.\\
  \part
  \begin{oneparcheckboxes}
    \choice T
    \choice F
  \end{oneparcheckboxes}
  Both Scheme and Python use dynamic typing.\\
  \part
  \begin{oneparcheckboxes}
    \choice T
    \choice F
  \end{oneparcheckboxes}
  A declarative language like Prolog allows us to describe how we solve a problem rather than what the answer will look like.\\
  \part
  \begin{oneparcheckboxes}
    \choice T
    \choice F
  \end{oneparcheckboxes}
  Prolog is a statically typed language.\\
  \part
  \begin{oneparcheckboxes}
    \choice T
    \choice F
  \end{oneparcheckboxes}
  Prolog's functionality is based on lambda calculus.\\
  \part
  \begin{oneparcheckboxes}
    \choice T
    \choice F
  \end{oneparcheckboxes}
  The initial rise in popularity of Object Oriented langugages was related to the sales of home personal computes and the need for GUIs.\\
  \part
  \begin{oneparcheckboxes}
    \choice T
    \choice F
  \end{oneparcheckboxes}
  In the model-view-controller design pattern, the controller implements the event handling.\\
  \part
  \begin{oneparcheckboxes}
    \choice T
    \choice F
  \end{oneparcheckboxes}
  Java's automatic garbage collection makes heap overflows impossible, unlike C++ where objects had to be explicitly deconstructed.\\
  \part
  \begin{oneparcheckboxes}
    \choice T
    \choice F
  \end{oneparcheckboxes}
  When a thread is waiting for a condition that will never happen, it is said to be deadlocked.\\
  \part
  \begin{oneparcheckboxes}
    \choice T
    \choice F
  \end{oneparcheckboxes}
  Concurrent programs provide performance improvements even on single processor machines.
 \end{parts}
 
 \newpage
 
 \question[8]
 Consider the four programming paradigms and the languages we covered this semester:
 \begin{center}
  \begin{enumerate}
   \item Imperative, Go
   \item Object-Oriented, Java
   \item Functional, ML
   \item Logical, Prolog
  \end{enumerate}
 \end{center}
  
  For each of the following problems, argue which paradigm is best suited to solve using the language listed above. Make sure you justify your choice. You will answer each paradigm/language pair only once. \textit{Hint: this is not a subjective question. If in doubt, pick the choice you think the professor would choose.}\\\\
  
  \begin{parts}
   \part
   A program to implement the AI of a character in a virtual world.
   \vspace*{\stretch{1}}
   
   \part
   A graphical user interface for architecture design software
   \vspace*{\stretch{1}}
   
   \part
   A program to handle dormitory room assignments for a large university given a set of room preferences
   \vspace*{\stretch{1}}
   
   \part
   A program that crawls through web page meta data to construct search engine information
   \vspace*{\stretch{1}}
   
  \end{parts}

 \newpage

\begin{center}
\begin{tabular}{|c|c||c|c||c|c|}
\hline 
(a) & argument & (i) & event-driven program & (q) & parameter\tabularnewline
 &  &  &  &  & \tabularnewline
(b) & activation record & (j) & functor & (r) & parallel program\tabularnewline
 &  &  &  &  & \tabularnewline
(c) & atomic proposition & (k) & garbage & (s) & race condition\tabularnewline
 &  &  &  &  & \tabularnewline
(d) & backtracking & (l) & horn clause & (t) & resolution\tabularnewline
 &  &  &  &  & \tabularnewline
(e) & concurrent program & (m) & lambda expression & (u) & run-time stack\tabularnewline
 &  &  &  &  & \tabularnewline
(f) & currying & (n) & lazy evaluation & (v) & semaphore\tabularnewline
 &  &  &  &  & \tabularnewline
(g) & deadlock & (o) & model-view-controller & (w) & thread\tabularnewline
 &  &  &  &  & \tabularnewline
(h) & eager evaluation & (p) & monitor & (x) & unification\tabularnewline
\hline 
\end{tabular}
\end{center}
\vspace{3em}
 
 \question[15]
 Match the term (a -- x) to the definition. Not all terms will be used.
 \\\\\\
 \begin{center}
 \begin{tabular}{|c|l|}
\hline 
~~~~term~~~~ & definition\tabularnewline
\hline 
 & \tabularnewline
 & an integer variable and an associated thread queue\tabularnewline
 & \tabularnewline
\hline 
 & \tabularnewline
 & memory model for keeping track of function calls and returns\tabularnewline
 & \tabularnewline
\hline 
 & \tabularnewline
 & block of heap memory that cannot be accessed by the program\tabularnewline
 & \tabularnewline
\hline 
 & \tabularnewline
 & contains a head and a list of predicates\tabularnewline
 & \tabularnewline
\hline 
 & \tabularnewline
 & a program designed to have two or more execution contexts\tabularnewline
 & \tabularnewline
\hline 
\end{tabular}
\end{center}

\newpage

Match the term (a -- x) to the definition (\textit{continued})

\begin{center}
 \begin{tabular}{|c|l|}
\hline 
~~~~term~~~~ & definition\tabularnewline
\hline 
 & \tabularnewline
 & a program unit that can be executed concurrently\tabularnewline
 & \tabularnewline
\hline 
 & \tabularnewline
 & concept particularly useful in event-driven GUI programs\tabularnewline
 & \tabularnewline
\hline 
 & \tabularnewline
 & a function that is not bound to an identifier\tabularnewline
 & \tabularnewline
\hline 
 & \tabularnewline
 & a program designed to have multiple threads active on different processors\tabularnewline
 & \tabularnewline
\hline 
 & \tabularnewline
 & encapsulate a shared variable with operations signal and wait\tabularnewline
 & \tabularnewline
\hline 
 & \tabularnewline
 & a variable identifier that appears in a function header\tabularnewline
 & \tabularnewline
\hline 
 & \tabularnewline
 & a declarative sentence that cannot be broken into simpler sentences\tabularnewline
 & \tabularnewline
\hline 
 & \tabularnewline
 & algorithm for abandoning the failed part of an attempted solution\tabularnewline
 & \tabularnewline
\hline 
 & \tabularnewline
 & translation of multiple argument function to an equivalent series
of single argument functions\tabularnewline
 & \tabularnewline
\hline 
\end{tabular}
\end{center}

\newpage

\question[10]
Consider this code snippet in some C-like language. Answer the questions below.\\
Assume indexing starts at 0.

\begin{lstlisting}
 // global variables
 int a = 1;
 int b = 2;
 void foo(int[] list, int i, int j) {
    i = i + 1;
    j = j + 1;
    list[i] = list[i] + j;
    print(a);
    print(b);
    print(list);
 }
 void main() {
    int x[5] = {5, 9, 7, 1, 5};
    foo(x, a, b);
    print(a);
    print(b);
    print(x);
 }
\end{lstlisting}

\begin{parts}
 \part
 In the above code, circle all function \textbf{parameters}.
 
 \part
 In the above code, draw a rectangle around all function \textbf{arguments}.
 
 \part
 Which values are printed if the language passes by value?
 \vspace*{\stretch{1}}
 
 \part Which values are printed if the language passes by value-result?
 \vspace*{\stretch{1}}
 
 \part Which values are printed if the language passes by reference?
 \vspace*{\stretch{1}}
\end{parts}

\newpage

\question[15]
The following questions deal with functional programming in teh Scheme language. You may use any of the primitive Scheme functions, such as \verb|if, cond, null?, equal?, eq?, eqv?, car, cdr and cons,| \verb|append, list?,| and \verb|member?|

\begin{parts}
 \part Define a Scheme function, \verb|odds|, that takes a list and returns every other element, starting with the first.
 \\\\
 Example use:
 \begin{lstlisting}{lisp}
  > (odds '(a b c d e f g))
  (a c e g)
  
  
  ; Write your function here
  (define (odds lst)
  
  
  
  
  
  
  
  
  
  
  ); end define
 \end{lstlisting}

\part
Define a Scheme function, \verb|evens|, that takes a list and returns the elements in even positions, starting with the second. You may use your function odds, if useful.
\\\\
Example use:
\begin{lstlisting}{lisp}
 > (evens '(a b c d e f g))
 (b d f)
 
 
 ; Write your function here
 (define (evens l)
 
 
 
 
 
 
 
 
 
 
 
 ); end define
\end{lstlisting}

\newpage

\part
Consider these functions. You should assume \verb|odds| and \verb|evens| function correctly even if you did not complete parts (a) and (b).
\\\\
\begin{lstlisting}{lisp}
 (define (plusOne x)
      (+ 1 x))
      
      
 (define (plusTwo x)
      (+ 2 x))
      
      
 (define (mystery lst)
      (if (null? lst)
      '()
      (append (map plusTwo (odds lst))
	   (map plusOne (evens lst))
      )
 ))
\end{lstlisting}

What is the output of this line of Scheme code?
\\
\begin{lstlisting}{lisp}
 > (mystery '(1 2 3 4 5 6))
\end{lstlisting}



\end{parts}

\newpage

\question[10]
What do the following Scheme expressions return?
\begin{parts}
 \part
 \verb|(car '(1 (2) 3 4 (5 6 (7 8)) 9))|
 \vspace*{\stretch{1}}
 
 \part
 \verb|(cdr '(1 (2) 3 4 (5 6 (7 8)) 9))|
 \vspace*{\stretch{1}}
 
 \part
 \verb|(caadr '(1 (2) 3 4 (5 6 (7 8)) 9))|
 \vspace*{\stretch{1}}
 
 \part
 \verb|(cddddr '(1 (2) 3 4 (5 6 (7 8)) 9))|
 \vspace*{\stretch{1}}
 
 \part
 \verb|(caar (cddddr '(1 ((2) 4) 3 4 (5 6 (7 8)) 9)))|
 \vspace*{\stretch{1}}
 
 \part
 \verb|(caddar (cddddr '(1 ((2) 4) 3 4 (5 6 (7 8)) 9)))|
 \vspace*{\stretch{1}}
\end{parts}

\newpage

\question[12]
Complete the following Prolog definitions.
\\\\
\textit{Some Hints:}
\begin{itemize}
 \item Underscore (\_) is an anonymous name (i.e., don't care).
 \item \verb=[A | B]= splits a list into head A and tail B
 \item \verb=[A, B | C]= takes first two elements A and B, and sets the tail C
 \item Bang (!) is the cut operator. E.g., \verb=someRule(A, B) :- !=. stops the recursion.
\end{itemize}

\begin{parts}
 \part
 Define \verb|last_but_one/2| function that finds the 2nd-to-last element of a list.
 \begin{lstlisting}
  % last_but_one(X, L) :- X is the last but one element of the list L
 \end{lstlisting}
 \vspace*{\stretch{1}}
 
 \part
 Define \verb|get_length/2| function that finds the number of elements of a list.
 \begin{lstlisting}{prolog}
  % get_length(L, N) :- the list L contains N elements
 \end{lstlisting}
 \vspace*{\stretch{1}}
 
 \part
 Define \verb|reverse_list/2| function that reverses a list. You may find a helper function to be useful.
 \begin{lstlisting}{prolog}
  % reverse_list(L1, L2) :- L2 is the list obtained from L1 by reversing
 \end{lstlisting}
 \vspace*{\stretch{1}}
 
 \part
 Define \verb|is_palindrome/1| function that determines if a list is a palindrome.
 \begin{lstlisting}{prolog}
  % is_palindrome(L) :- L is a palindrome list
 \end{lstlisting}
 \vspace*{\stretch{1}}
 
 \end{parts}

 \newpage
 
 \fullwidth{\begin{center}THIS PAGE INTENTIONALLY LEFT BLANK\end{center}}
 
\end{questions}
\end{document}
