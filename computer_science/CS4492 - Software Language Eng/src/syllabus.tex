%% LyX 2.0.5.1 created this file.  For more info, see http://www.lyx.org/.
%% Do not edit unless you really know what you are doing.
\documentclass[english]{paper}
\usepackage[T1]{fontenc}
\usepackage[latin9]{inputenc}
\usepackage[letterpaper]{geometry}
\geometry{verbose,tmargin=0.75in,bmargin=0.75in,lmargin=0.75in,rmargin=0.75in}
\usepackage{array}
\usepackage{multirow}
\usepackage{listings}
\usepackage[table]{xcolor}
\usepackage{enumitem,multicol}
\usepackage[colorlinks=true]{hyperref}
\usepackage{tikz}
\usepackage{tcolorbox}
\usepackage{lmodern}
\newcommand*\circled[1]{\tikz[baseline=(char.base)]{%
			\node[shape=circle,fill=blue!20,draw,inner sep=2pt] (char) {#1};}}

\makeatletter

%%%%%%%%%%%%%%%%%%%%%%%%%%%%%% LyX specific LaTeX commands.
%% Because html converters don't know tabularnewline
\providecommand{\tabularnewline}{\\}

\makeatother

\usepackage{babel}
\begin{document}

{\noindent\begin{center}\Large CSCI 305\\Concepts of Programming Languages\\Spring 2018\end{center}}

\begin{tcolorbox}
\noindent\textbf{Class meeting time:} MWF 8:00--8:50am, Roberts 101\\
\textbf{Instructor:} Isaac Griffith\\
\textbf{Contact:} isaac.griffith@msu.montana.edu\\
\textbf{Office Hours:} MWF, 0900--1000, EPS 254\\
\textbf{Office Location:} Barnard Hall Room \#355\\
\textbf{Final Exam:} May 2, 2018 from 08:00 to 09:50 in Roberts 101
\end{tcolorbox}

\section*{Required Textbook}
Modern Programming Languages, A.B. Webber, Franklin Beedle and Associates Inc.

\section*{Optional Textbook}
Concepts of Programming Languages (11th Edition), R.W. Sebasta, Pearson

\section*{Pre-requisites}
\begin{itemize}
  \item CSCI 132 Basic Data Structures and Algorithms
  \item CSCI 246 Discrete Mathematics
\end{itemize}

\section*{Course Description}

This course is an examination of the constructs of programming languages, the four major programming paradigms, and several programming languages, 
including their application and underlying execution models. The paradigms examined in this course include object-oriented, functional, and logical. 
Students will gain exposure to and experience in a variety of languages, including ML, Prolog, and Java. Several other languages will be briefly
explored through a series of short lectures. 

\section*{Course Objectives}

At the end of this course, the student will be able to:
\begin{itemize}
 \item Understand the history of programming and the development of languages
 \item Interpret and develop specifications for programming languages
 \item Understand different models for programming and for associated languages
 \item Understand the relationship between system architectures and programming languages
 \item Understand the relationship between programming language features and how the language is translated
 \item Understand how language designers take different design approaches based on the language goals
 \item Gain practical experience with several new programming languages
 \item Be prepared to take CSCI 468, Compilers
\end{itemize}

\section*{Instruction}
Course schedule, lecture slides, and assignments are available through the course website.

\section*{Grading Policy}
Grades will be based on participation, three homeworks, six programming labs, two midterm exams, and a final exam.

\section*{Attendance and Participation}
Class attendance is mandatory and participation is highly encouraged. You are responsible for all the material covered in class. Prepare in advance for class by reading and styding the assigned material, and by making sure you unerstand the previous lecture. Thre will be six in-class Participation Excercises given on random dates throughout the semester. The lowest quiz score will be dropped and the remaining five scores will count towards 10\% of your course grade.

\section*{Students with Disabilities}
A student who desires accommodation for a disability must submit appropriate documentation of the disability and request for accommodations to The Office 
of Disability, Re-Entry, and Veteran's Services Room \#155 Strand Union Building, (406) 994-2824.

\section*{Academic Integrity}
Honesty and integrity is expected in all class work. The standards set by Montana State Univerity's academic integrity and student conduct guidelines apply
to this class (http://shiva.msu.montana.edu/courses/conduct.html). Academic misconduct is unacceptable. It is the responsibility of all students to adhere
to strict standards of integrity in their professional and scholarly activities. Misconduct will be treated swiftly and harshly.

It is a breach of academic integrity to present the ideas or works of another as one's own work, or to permit another to present one's work without customary
and proper acknowledgement of authorship. Students may collaborate with other students only as expressly permitted by the instructor. Students are responsible
for the honest completion and representation of their work, the appropriate citation of sources and the respect and recognition of others' academic endeavors.
According to Montana State University Conduct Guidelines and Grievance Procedures for STudents, academic misconduct includes cheating, plagiarism, forgery,
facilitation or aiding academic dishonesty; multiple submission, theft of instructional materials or tests; unauthorized access to, manipulation of or tampering
with laboratory equipment, experiments, or computer programs, without proper authorization; alteration of grades or files; misuse of research data in reporting
results; use of personal relationships to gain grades or favors, or otherwise attempting to obtain grades or credit through fraudulent means.

\section*{Policy on Cell Phones, Laptops, and Tablets}

Bringing your personal phones, tables or laptops to class is fine. I do require that you set your phones to at least silent, and mute any sounds from either your tablet or laptop. Also refrain from replying or sending texts, social media, or phone calls during class, if you feel that you must do this then please leave the class and take care of it in the hallway. Further, playing games in class will also not be tolerated. The distractions that these actions cause both for your classmates as well as for your instructor are taxing on the limited time we have. If you are found to be performing any of these actions you will be asked to leave the classroom for the day.
\end{document}