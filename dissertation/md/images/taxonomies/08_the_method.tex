\chapter{Putting It All Together: The Method}

\section{Introduction}
The work documented in Chapters \ref{chap:dpd} through \ref{chap:detect} have
illustrated the underlying components of a much larger methodology. In this
chapter we will illustrate this method, how it will be used to solve the
problems outlined in \ref{chap:intro} and its possible future implications. We
begin with the method itself.

\section{The Method}


\section{Using the Method}


\section{Future Implications and Conclusion}
This chapter illustrated in detail the underlying research method embodied
within this work and the experimental Arc Framework. Through the use of the
developed workflows in the Arc system the following chapters will show the power
of this method to explore software engineering phenomena, specifically
concerning design pattern grime (which is the specific subject of this current
research). Yet, this method is not reduced to understanding a single type of
phenomena, nor a class (such a static issues), rather it has the ability to be
used to study a vast array of software engineering issues, as this is the
embodiment of the scientific process.

The implications of this method is that it can extend to any form of design
disharmony which manifests itself within the confines of software and software
artifacts. Such issues include (but are not limited to) other technical debt
issues such as AntiPatterns, Code Smells, Static Analysis issues. With an
extension to the Arc Framework, behavioral and dyanmic analysis issues are well
within the grasp of study, as well. Furthermore, issues related to security are
also easily studied. All of this boils down to the fact that if it can be
described in an injection, we can then begin to study the problem.

Studies beyond design disharmonies are also expected, such as evaluating the
effects of refactoring. But the limits are not set at source code level.
Evaluating such artifacts as requirments, documentation, and even the effects of
software engineering team structures, and developer interactions can and should
be studied using such a method. The implications of this method is that it
provides a sound basis for experimenting within the confines of a software
system, allowing for the formation of a strong empirical base upon which case
studies and longitudinal studies can the be situated. Thus allowing for the
interrelation of multiple studies and the growth of a large body of knowledge.
