\documentclass[addpoints]{exam}
\usepackage[usenames,dvipsnames]{xcolor}
\usepackage{titling}
\usepackage{listings}
\usepackage{multicol}
\usepackage[letterpaper]{geometry}
\geometry{verbose,tmargin=0.75in,bmargin=0.75in,lmargin=0.75in,rmargin=0.75in}

\newcommand{\subtitle}[1]{%
  \posttitle{%
    \par\end{center}
    \begin{center}\large#1\end{center}
    \vskip0.5em}%
}

\lstset{%set Code listings styles
	language=Java, % program language for keywords and comments styles
	basicstyle=\small, %font size and style
	identifierstyle=\color{black}, %variable name style
	stringstyle=\ttfamily, %string style
	keywordstyle=\color{blue}\bfseries, %language keyword style
	commentstyle=\color{blue!75}\itshape, %commentstyle
	breaklines=true,  % sets automatic line breaking
	breakatwhitespace=false,   %break line not just at whitespaces
	showspaces=false,
	showstringspaces=false,
}

\pagestyle{headandfoot}
\runningheadrule
\firstpageheadrule
\firstpageheader{CSCI 305}{March 6, 2018 -- Midterm Exam}{Name:\enspace\hspace{1.5in}}
\runningheader{CSCI 305}{March 6, 2018 -- Midterm Exam}{Name:\enspace\hspace{1.5in}}
\firstpagefooter{}{Page \thepage\ of \numpages}{}
\runningfooter{}{Page \thepage\ of \numpages}{}
% \printanswers

\begin{document}
\begin{coverpages}
\begin{center}
  {\makebox[3.5in]{\hrulefill}\\
  \vspace{1.5em}
  {\Large MONTANA STATE UNIVERSITY}\\
  \vspace{.5em}
  PROGRAMMING LANGUAGES\\
  \vspace{.5em}
  Midterm\\
  \vspace{.5em}
  SPRING, 2018
  \\
  \vspace{1em}
  \makebox[3.5in]{\hrulefill}\\
  \par}
  \vspace{0.5in}
  \makebox[3.5in]{Name:\enspace\hrulefill}

  \vspace{0.2in}

  \makebox[3.5in]{Student Id:\enspace\hrulefill}
  \vspace{0.2in}  

  \setlength{\doublerulesep}{0.1in}
  \gradetable[v][questions]
\end{center}

\end{coverpages}

\begin{questions}
 \question[30]
 Mark each question as either True (T) or False (F).\\
 \begin{parts}
  \part
  \begin{oneparcheckboxes}
    \choice T
    \choice F
  \end{oneparcheckboxes}
  C++ is considered by many to be the first object-oriented programming language.\\
  \part
  \begin{oneparcheckboxes}
    \choice T
    \choice F
  \end{oneparcheckboxes}
  COBOL was designed as a programming language for business applications.\\
  \part
  \begin{oneparcheckboxes}
    \choice T
    \choice F
  \end{oneparcheckboxes}
  LISP was originally developed for Artificial Intelligence problems.\\
  \part
  \begin{oneparcheckboxes}
    \choice T
    \choice F
  \end{oneparcheckboxes}
  A strongly typed programming language is one that detects all type errors at either compile time or run time.\\
  \part
  \begin{oneparcheckboxes}
    \choice T
    \choice F
  \end{oneparcheckboxes}
  A non-terminal that occurs on the far left of any production rule is left associative.\\
  \part
  \begin{oneparcheckboxes}
    \choice T
    \choice F
  \end{oneparcheckboxes}
  The ``Von Neuman'' architecture is still used for the basis of most computers today.\\
  \part
  \begin{oneparcheckboxes}
    \choice T
    \choice F
  \end{oneparcheckboxes}
  Any finite language can be defined by a regular expression.\\
  \part
  \begin{oneparcheckboxes}
    \choice T
    \choice F
  \end{oneparcheckboxes}
  A grammar with a finite set of terminal symbols can only define a finite language.\\
  \part
  \begin{oneparcheckboxes}
    \choice T
    \choice F
  \end{oneparcheckboxes}
  EBNF was designed to make context free grammars easier to read. It does not offer any more descriptive power than BNF.\\
  \part
  \begin{oneparcheckboxes}
    \choice T
    \choice F
  \end{oneparcheckboxes}
  Associativity rules only apply to operators of the same precedence.\\
  \part
  \begin{oneparcheckboxes}
    \choice T
    \choice F
  \end{oneparcheckboxes}
  The order of production rules is not significant, i.e., two grammars with identical rules but given in a different order will always define the same language.\\
  \part
  \begin{oneparcheckboxes}
    \choice T
    \choice F
  \end{oneparcheckboxes}
  Overloading functions and operators can only be done in dynamically typed languages because of their greater flexibility.\\
  \part
  \begin{oneparcheckboxes}
    \choice T
    \choice F
  \end{oneparcheckboxes}
  An advantage of using an interpreted language is that it runs faster than compiled code.\\
  \part
  \begin{oneparcheckboxes}
    \choice T
    \choice F
  \end{oneparcheckboxes}
  A programming language that uses dynamic scoping allows programs to see every symbol table on the stack.\\
  \part
  \begin{oneparcheckboxes}
    \choice T
    \choice F
  \end{oneparcheckboxes}
  The procedural programming paradigm treats procedures as first class objects.
 \end{parts}

 \newpage

\question[2]
When two variable identifiers are allowed to share the same address pointer it is known as:\\
\begin{choices}
 \choice dynamic binding
 \choice dynamic scoping
 \choice aliasing
 \choice weak typing
 \choice dynamic typing
\end{choices}

\vspace{2em}

\question[2]
Context free grammars are used to capture which aspect of a programming language?\\
\begin{choices}
 \choice lexical structure
 \choice syntax
 \choice type system
 \choice operational semantics
 \choice axiomatic semantics
\end{choices}

\vspace{2em}

\question[2]
Choose the (one) example of orthoganality from the statements below\\
\begin{choices}
 \choice An array element can be any data type except void or a function.
 \choice All statements (including assignment, if, and while) return some value.
 \choice Parameters are passed by value, unless they are arrays, in which case they are pased by reference.
 \choice Structures (but not arrays) may be returned from a function.
 \choice Function passing is not permitted.
\end{choices}

\vspace{2em}

\question[2]
In Java, a method declared in an interface is implicitly:\\
\begin{choices}
 \choice static
 \choice private
 \choice final
 \choice abstract
 \choice none of the above
\end{choices}

\vspace{2em}

\question[2]
Circle all of the following that are Object Oriented languages.\\
\begin{choices}
 \choice C\#
 \choice Fortran 77
 \choice Java
 \choice LISP
 \choice Perl
\end{choices}

\newpage

\question
Consider the following code in some C-style language. Assume the \verb|writeln| function prints to standard output followed by an \verb|EOL|.

\begin{lstlisting}
 // global variables
 int foo = 3;
 int bar = 5;
 
 void printmult(int bar) {
     writeln foo * bar;
 }
 
 void addbar(int bar) {
     foo = foo + bar;
     printmult(foo);
 }
 
 void test() {
     int foo = 2;
     int bar = 7;
     printmult(4);
     
     foo = 1;
     bar = 13;
     addbar(9);
     writeln foo;
 }
\end{lstlisting}

\begin{parts}
 \part[5]
 What does the function \verb|test| print if the language uses static scoping?
 \vspace*{\stretch{1}}
 
 \part[5]
 What does the function \verb|test| print if the language uses dynamic scoping?
 \vspace*{\stretch{1}}
 
\end{parts}

\newpage

\question
Answer these questions about the following grammar. The starting symbol is \verb|<re>|.\\

\begin{verbatim}
 <do> ::= <me>
 <do> ::= <do> % <me>
 <re> ::= <do> & <re>
 <re> ::= <do>
 <me> ::= ( <re> )
 <me> ::= a | b
\end{verbatim}
\vspace{2em}

\begin{parts}
 \part[2]
 Which operator has higher precedence?
 \vspace*{\stretch{1}}
 
 \part[2]
 What is the associativity of the \verb|%| operator?
 \vspace*{\stretch{1}}
 
 \part[2]
 What is the associativity of the \verb|&| operator?
 \vspace*{\stretch{1}}
 
 \part[2]
 Does this grammar describe a finite or an infinite language?
 \vspace*{\stretch{1}}
 
 \part[2]
 Is this grammar ambiguous?
 \vspace*{\stretch{1}}
 
 \part[5]
 Draw a parse tree to explain the string:
 \begin{center}\verb|b & a % (a & b) & a|.\end{center}
 \vspace*{\stretch{1}}
\end{parts}

\newpage

\question[15]
complete the following Java loop skeleton below. Your code should match street addresses of a certain format and print out the information in the manner described below.

Using regular expressions, match street addresses of the format: a number followed by one or more words and ending in the string ``st'', ``dr'', or ``ave''. For example, your regex should be able to return ``1212 pine st''. You do not need to worry about all of the other possible road types, just the three listed above.

The following is an example input:

\begin{verbatim}
      Thomas Aquinas, (555) 555-5555, 1212 Pine St Bozeman, MT 59715
      Bram Stoker, (123) 456-7890, 123 W College Dr, Helena, MT 59601
      Ben Kingsley, (321) 123-6756, 9876 N 13th Ave, Nome, AL 99762
\end{verbatim}

\begin{lstlisting}
 // This loop processes some text file line by line
 // You only need to complete the code within the loop
 
 while(my $line = <STDIN>) {
 
      // Coverts all text to lower case
      $line = lc($line);
\end{lstlisting}
\vspace*{\stretch{1}}
\begin{lstlisting}
 } // End while loop
\end{lstlisting}

\newpage

\question[20]
Consider the following Java code:
\begin{lstlisting}
 interface Inter {
    int fun();
 }
 interface Face {
    int work();
 }
 abstract class Picaso {
    int blue = 42;
    String period() { return "."; }
    abstract int paint();
 }
 class Reach extends Picaso implements Inter {
    public int paint() { return 3; }
    public int fun() { return 5; }
    int up(int input) { return input * 2; }
 }
 class Arm extends Reach implements Face {
    Arm(int wave) { blue = wave; }
    public int work() { return 7; }
    public int fun() { return blue + 2; }
    String up(String input) { return input + period(); }
 }
\end{lstlisting}

Now consider this concrete implementation of two classes:
\begin{lstlisting}
 public static void main(String[] args) {
    Reach r1 = new Reach();
    Reach r2 = new Arm(10);
    
    // Expressions executed here
 }
\end{lstlisting}

Give the value for each expression below or ``error'' if an error would occur.
\begin{multicols}{2}
 \begin{parts}
  \part
  r1.fun()
  \vspace*{0.5in}
  
  \part
  r2.fun()
  \vspace*{0.5in}
  
  \part
  r1.blue
  \vspace*{0.5in}
  
  \part
  r2.blue
  \vspace*{0.5in}
  
  \part
  r2.work()
  \vspace*{0.5in}
  
  \part
  r2.paint()
  \vspace*{0.5in}
  
  \part
  r1.up(3)
  \vspace*{0.5in}
  
  \part
  r2.up(12)
  \vspace*{0.5in}
  
  \part
  r2.up(``7'')
  \vspace*{0.5in}
  
  \part
  r1.up(``16'')
  \vspace*{0.5in}
  
 \end{parts}

\end{multicols}

\newpage

\fullwidth{\begin{center}THIS PAGE INTENTIONALLY LEFT BLANK\end{center}}

% \fullwidth{\begin{center}\textbf{EXTRA CREDIT}\end{center}}
% \renewcommand{\thequestion}{EC\arabic{question}}
% \question
% Consider this Java code. What is the output of this program?
% 
% \begin{lstlisting}
%  int x = 0;
%  for (int i = 0; i < 5; i++) {
%      x = x++;
%      System.out.print(x);
%  }
% \end{lstlisting}
% \vspace*{\stretch{1}}
% 
% \question
% Consider this C code snippet. Assuming \verb|a| and \verb|b| have been previously assigned values what does this line accomplish?
% \begin{center} \verb|a = (a + b) - (b = a)| \end{center}
% \vspace*{\stretch{1}}
% 
% \question
% Consider this C code snippet. If the value of \verb|a| is 0, what is the value of \verb|c| following this statement?
% \begin{center} \verb|c = !a?10:100;| \end{center}
% \vspace*{\stretch{1}}
 
\end{questions}
\end{document}
