\documentclass{article}
\usepackage[colorlinks=true]{hyperref}
\usepackage[usenames,dvipsnames]{xcolor}
\usepackage{titling}
\usepackage{listings}
\usepackage{enumitem,multicol}
\usepackage{lmodern}
\usepackage{tcolorbox}
\usepackage{tikz}
\newcommand*\circled[1]{\tikz[baseline=(char.base)]{%
			\node[shape=circle,fill=blue!20,draw,inner sep=2pt] (char) {#1};}}
\usepackage[letterpaper]{geometry}
\geometry{verbose,tmargin=0.75in,bmargin=0.75in,lmargin=0.75in,rmargin=0.75in}

\newcommand{\subtitle}[1]{%
  \posttitle{%
    \par\end{center}
    \begin{center}\large#1\end{center}
    \vskip0.5em}%
}

\lstset{%set Code listings styles
	language=Prolog, % program language for keywords and comments styles
	basicstyle=\small, %font size and style
	identifierstyle=\color{black}, %variable name style
	stringstyle=\ttfamily, %string style
	keywordstyle=\color{blue}\bfseries, %language keyword style
	commentstyle=\color{OliveGreen}\itshape, %commentstyle
	breaklines=true,  % sets automatic line breaking
	breakatwhitespace=false,   %break line not just at whitespaces
	showspaces=false,
	showstringspaces=false,
}

\title{``Prolog Parentage''}
\subtitle{Prolog Programming Lab}
\author{Concepts of Programming Languages\\CSCI 305, Spring 2018}
\date{Due: 3/25/2015 @ 12:30 PM}

\begin{document}

\maketitle

\section*{Prolog}
For this lab, you will use Prolog. I expect you to use the prolog installed in the VM provided for the course, but you can also install it on Windows, Mac, and Linux by downloading from here: \url{http://www.swi-prolog.org/download/stable}. Prolog is also installed on the departmental lab Linux machines.

\section*{Dataset}
Begin by downloading the file \verb|royal.pro| form BrightSpace\footnote{This data was adapted from \url{http://ftp.aset.psu.edu/~saw/royal/}}. This file is a set of Prolog facts that represents the genealogy of the royal House of Windsor.

\section*{Getting Started}
Next, create a new file \verb|[lastname]_[firstname].prolog2.pro|. This file should be in the same directory as \verb|royal.pro|. In this file, you will include all your Prolog rules you will be asked to write in the lab.
\\\\
Your program should begin with your name as a comment.
\begin{lstlisting}
 % First_name Last_name
 % CSCI 305 Prolog Lab 2
\end{lstlisting}

\section*{Warmup}
Check that everything is working. First load the \verb|royal.pro| file by choosing File$\rightarrow$Consult and choosing the file. Next, load the file you created above in SWI-Prolog.
\\\\
\noindent Now, run the query:

\begin{lstlisting}
 ?- parent(X, 'Queen Elizabeth II').
\end{lstlisting}

\noindent The symbol \verb|?-| is the Prolog query prompt. You do not need to retype this. This query asks who is the parent of Queen Elizabeth II. Prolog first returns the result \verb|X = 'King George VI'|. Type a semicolon (;). This will prmpt Prolog for the next result. Prolog then returns \verb|X = 'Lady Elizabeth Bowes-Lyon'.| The period following teh second result indicates there are no additional results.
\\\\
\noindent The entire query and result interaction should appear as:
\begin{lstlisting}
 ?- parent(X, 'Queen Elizabeth II').
 X = 'King George VI' ;
 X = 'Lady Elizabeth Bowes-Lyon'.
\end{lstlisting}
Replicate this exchange on your own machine.

\section*{Parents}
In this lab you will create a number of Prolog rules that define a number of common familial relationships.
\\\\
\noindent In Prolog parlance, we define a rule as its name and its arity. For instance the rule \verb|mother/2| indicates a rule named \verb|mother| with two arguments.
\begin{lstlisting}
 mother(M,C):- parent(M,C), female(M).
\end{lstlisting}
\noindent Add this rule to your Prolog file. Reload your file. Test your new rule with the query:
\begin{lstlisting}
 ?- mother(X, 'Queen Elizabeth II').
\end{lstlisting}
\noindent which should return the single result:
\begin{lstlisting}
 X = 'Lady Elizabeth Bowes-Lynon'.
\end{lstlisting}
\noindent Now write the rule \verb|father/2| and add it to your file.


\section*{More Rules}
Now write a set of additional rules. In your rules, you are encouraged to make use of other rules you have written. This may effect the order you choose to implement these rules.
\\\\
\noindent Make sure you get the order of the arguments correct as switching then will change the meaning of the function.
\begin{itemize}
 \item \verb|spouse/2|
 \item \verb|child/2|
 \item \verb|son/2|
 \item \verb|daughter/2|
 \item \verb|sibling/2|
 \item \verb|brother/2|
 \item \verb|sister/2|
 \item \verb|uncle/2| \hspace{2em} two rules: one by blood, one by marriage.
 \item \verb|aunt/2|\,\,\, \hspace{2em} two rules: one by blood, one by marriage.
 \item \verb|grandparent/2|
 \item \verb|grandfather/2|
 \item \verb|grandmother/2|
 \item \verb|grandchile/2|
\end{itemize}

\noindent For the next two rules, make use of rules you have already written. It may take more than one rule to define these functions. You may find tracing useful. To debug, you can use the keyword \verb|trace| to enable tracing and \verb|notrace| to disable.
\\\\
\noindent Write the rules:
\begin{itemize}
 \item \verb|ancestor/2|\,\,\,\,\,\, \hspace{2em} \verb|ancestor(X, Y)| means \verb|X| is the ancestor of \verb|Y|
 \item \verb|descendant/2| \hspace{2em} \verb|descendant(X, Y)| means \verb|X| is the descendant of \verb|Y|
\end{itemize}

\section*{Numbers in Prolog}
Now we will write some Prolog rules that require numeric comparisons. Write the rules:
\begin{itemize}
 \item \verb|older/2|
 \item \verb|younger/2|
\end{itemize}

\noindent The rule \verb|older(X, Y)| indicates person \verb|X| is older than person \verb|Y|. The rule \verb|younger/2| should be written in a similar manner.
\\\\
\noindent Lastly, write the rule \verb|regentWhenBorn/2|. This rule, \verb|regentWhenBorn(X, Y)|, should ask who was King or Queen (X) when person Y was born.
\\\\

\begin{tcolorbox}[sharp corners]
 \section*{Extra Credit (10 pts)}
 Write rule(x) that define the first cousin relationship, \verb|cousin/2|. You should look up the formal definition of cousin. You do not need to support second or third cousins, nor cousins removed.
 \\\\
 \noindent \textbf{Question EC1:} What is the result of query:
 \begin{lstlisting}
  ?- cousin(X, 'Charles, Prince of Wales').
 \end{lstlisting}
 \noindent \textbf{Question EC2:} What is the result of query:
 \begin{lstlisting}
  ?- cousin('Charles, Prince of Wales', X).
 \end{lstlisting}
 \noindent Enable tracing and run both of these queries again.
 \\\\
 \noindent \textbf{Question EC3:} Which resulted in more steps? Explain why. Provide both of these traces in a file \verb|[lastname]_[firstname].prolog2.ec.trace.txt|.
\end{tcolorbox}

\begin{tcolorbox}[sharp corners]
 \section*{Lab Questions}
 For these lab questions, you are expected to use Prolog to answer. Answers that are given without providing the Prolog rules that derive it will receive no credit. Some queries have multiple answers and you are expected to provide all results to the lab questions. Use the semicolon (;) to prompt Prolog for additional results. However, you may omit duplicate results, if the same name appears more than once in your result set.
 \\\\
 \noindent \textbf{Question 1:} What is the result of query: \verb|?- father(X, 'Queen Elizabeth II').|?
 \\\\
 \noindent \textbf{Question 2:} What is the result of query: \verb|?- grandmother(X, 'Queen Elizabeth II').|?
 \\\\
 \noindent \textbf{Question 3:} What is the result of query: \verb|?- grandfather(X, 'Queen Elizabeth II').|?
 \\\\
 \noindent \textbf{Question 4:} What is the result of query: \verb|?- grandparent(X, 'Queen Elizabeth II').|?
 \\\\
 \noindent \textbf{Question 5:} What is the result of query: \verb|?- grandparent('Queen Elizabeth II', X).|?
 \\\\
 \noindent \textbf{Question 6:} What is the result of query: \verb|?- sibling(X, 'Queen Elizabeth II').|?
 \\\\
 \noindent \textbf{Question 7:} What is the result of query: \verb|?- son(X, 'Queen Elizabeth II').|?
 \\\\
 \noindent \textbf{Question 8:} What is the result of query: \verb|?- daughter(X, 'Queen Elizabeth II').|?
 \\\\
 \noindent \textbf{Question 9:} What is the result of query: \verb|?- aunt(X, 'Lady Diana Spencer').|?
 \\\\
 \noindent \textbf{Question 10:} What is the result of query:\\\verb|?- spouse(X, 'Prince William, Duke of Cambridge').|?
 \\\\
 \noindent \textbf{Question 11:} What is the result of query: \verb|?- ancestor(X, 'Prince Henry of Wales').|?
 \\\\
 \noindent \textbf{Question 12:} What is the result of query: \verb|?- descendant('Queen Victoria', Y).|?
 \\\\
 \noindent \textbf{Question 13:} What is the result of query:\\\verb|?- older('Prince Henry of Wales', 'Prince William, Duke of Cambridge').|?
 \\\\
 \noindent \textbf{Question 14:} What is the result of query: \verb|?- older(X, 'Queen Elizabeth II').|?
 \\\\
 \noindent \textbf{Question 15:} What is the result of query:\\\verb|?- regentWhenBorn(X, 'Queen Elizabeth II').|?
\end{tcolorbox}

\section*{Troubleshooting}
This lab requires an independent study of the Prolog language. You are encouraged to use any web tutorials and resources to learn Prolog. Given the size of the class, I will not be able to debug your code for you. Please do not send panicked emails requesting I fix your bug for you. Allow yourself plenty of time, and use patience, perseverance, and the internt to debug your code. I will gladly answer clarifying questions about the goals and instructions of the Lab assignment.
\\\\
\begin{tcolorbox}[sharp corners]
 \section*{Lab Questions}
 The following questions are for feedback and evaluation purposes. Points are awarded for any sincere answer.
 \\\\
 \noindent \textbf{Question 16:} Name something you like about Prolog. Explain.
 \\\\
 \noindent \textbf{Question 17:} Name something you dislike about Prolog. Explain.
 \\\\
 \noindent \textbf{Question 18:} Did you enjoy this lab? Which aspects did you like and/or dislike?
 \\\\
 \noindent \textbf{Question 19:} Approximately how many hours did you spend on this lab?
 \\\\
 \noindent \textbf{Question 20:} Do you think you will use Prolog again? For which type(s) of project(s)?
\end{tcolorbox}

\section*{Submission}
Each student will complete and submit this assignment individually or in a two-person team. Do not consult with others. However, you are encouraged to use the internet to learn any aspect of Prolog you need to complete the assignment, but not to answer the questions asked in this lab.
\\\\
Comment your program heavily. Intelligent comments and a clean, readable formatting of your code accounts for 20\% of your grade.
\\\\
Save the final version of your program and zip the source code into a file named \verb|[lastname]_[firstname].prolog2.zip|. Type your lab questions in plain text as \verb|[lastname]_[firstname].prolog2.txt|. Include your name in the text file.
\\\\
We must be able to run your program from the command line with no arguments.
\\\\
Submit your files to the Java Program 2 dropbox folder on BrightSpace. Submit your files before the due date as late submissions will not be accepted.

\end{document}